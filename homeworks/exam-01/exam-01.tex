\documentclass{article}

\usepackage{fancyhdr}
\usepackage{extramarks}
\usepackage{amsmath}
\usepackage{amsthm}
\usepackage{amsfonts}
\usepackage{tikz}
\usepackage[plain]{algorithm}
\usepackage{algpseudocode}
\usepackage{listings}
\usepackage{xcolor}
\usepackage[english]{babel}
\usepackage[T1]{fontenc}
\usepackage{lmodern,mathrsfs}
\usepackage{xparse}
\usepackage[inline,shortlabels]{enumitem}
\setlist{topsep=2pt,itemsep=2pt,parsep=0pt,partopsep=0pt}
\usepackage[dvipsnames]{xcolor}
\usepackage[utf8]{inputenc}
\usepackage[a4paper,top=0.5in,bottom=0.2in,left=0.5in,right=0.5in,footskip=0.3in,includefoot]{geometry}
\usepackage[most]{tcolorbox}
\tcbuselibrary{minted} % tcolorbox minted library, required to use the "minted" tcb listing engine (this library is not loaded by the option [most])
\usepackage{minted} % Allows input of raw code, such as Python code
% \usepackage[colorlinks]{hyperref}


\usetikzlibrary{automata,positioning}

\tcbset{
    pythoncodebox/.style={
        enhanced jigsaw,breakable,
        colback=gray!10,colframe=gray!20!black,
        boxrule=1pt,top=2pt,bottom=2pt,left=2pt,right=2pt,
        sharp corners,before skip=10pt,after skip=10pt,
        attach boxed title to top left,
        boxed title style={empty,
            top=0pt,bottom=0pt,left=2pt,right=2pt,
            interior code={\fill[fill=tcbcolframe] (frame.south west)
                --([yshift=-4pt]frame.north west)
                to[out=90,in=180] ([xshift=4pt]frame.north west)
                --([xshift=-8pt]frame.north east)
                to[out=0,in=180] ([xshift=16pt]frame.south east)
                --cycle;
            }
        },
        title={#1}, % Argument of pythoncodebox specifies the title
        fonttitle=\sffamily\bfseries
    },
    pythoncodebox/.default={}, % Default is No title
    %%% Starred version has no frame %%%
    pythoncodebox*/.style={
        enhanced jigsaw,breakable,
        colback=gray!10,coltitle=gray!20!black,colbacktitle=tcbcolback,
        frame hidden,
        top=2pt,bottom=2pt,left=2pt,right=2pt,
        sharp corners,before skip=10pt,after skip=10pt,
        attach boxed title to top text left={yshift=-1mm},
        boxed title style={empty,
            top=0pt,bottom=0pt,left=2pt,right=2pt,
            interior code={\fill[fill=tcbcolback] (interior.south west)
                --([yshift=-4pt]interior.north west)
                to[out=90,in=180] ([xshift=4pt]interior.north west)
                --([xshift=-8pt]interior.north east)
                to[out=0,in=180] ([xshift=16pt]interior.south east)
                --cycle;
            }
        },
        title={#1}, % Argument of pythoncodebox specifies the title
        fonttitle=\sffamily\bfseries
    },
    pythoncodebox*/.default={}, % Default is No title
}

% Custom tcolorbox for Python code (not the code itself, just the box it appears in)
\newtcolorbox{pythonbox}[1][]{pythoncodebox=#1}
\newtcolorbox{pythonbox*}[1][]{pythoncodebox*=#1} % Starred version has no frame

% Custom minted environment for Python code, NOT using tcolorbox
\newminted{python}{autogobble,breaklines,mathescape}

% Custom tcblisting environment for Python code, using the "minted" tcb listing engine
% Adapted from https://tex.stackexchange.com/a/402096
\NewTCBListing{python}{ !O{} !D(){} !G{} }{
    listing engine=minted,
    listing only,
    pythoncodebox={#1}, % First argument specifies the title (if any)
    minted language=python,
    minted options/.expanded={
        autogobble,breaklines,mathescape,
        #2 % Second argument, delimited by (), denotes options for the minted environment
    },
    #3 % Third argument, delimited by {}, denotes options for the tcolorbox
}


% Basic Document Settings
\topmargin=-0.45in
\evensidemargin=0in
\oddsidemargin=0in
\textwidth=6.5in
\textheight=9.0in
\headsep=0.25in
\linespread{1.1}

\pagestyle{fancy}
\lhead{\hmwkAuthorName}
\chead{\hmwkClass\ (\hmwkClassInstructor): \hmwkTitle}
\rhead{\firstxmark}
\lfoot{\lastxmark}
\cfoot{\thepage}
\renewcommand\headrulewidth{0.4pt}
\renewcommand\footrulewidth{0.4pt}
\setlength\parindent{0pt}

% Homework Details
\newcommand{\hmwkTitle}{Examen 2}
\newcommand{\hmwkDueDate}{Abril 10, 2025}
\newcommand{\hmwkClass}{ESMA 5015}
\newcommand{\hmwkClassInstructor}{Damaris Santana}
\newcommand{\hmwkAuthorName}{\textbf{Alejandro Ouslan}}

% Title Page
\title{
	\vspace{2in}
	\textmd{\textbf{\hmwkClass:\ \hmwkTitle}}\\
	\normalsize\vspace{0.1in}\small{Due\ on\ \hmwkDueDate}\\
	\vspace{0.1in}\large{\textit{\hmwkClassInstructor}}
	\vspace{3in}
}

\author{\hmwkAuthorName}
\date{}




% Begin document
\begin{document}
\maketitle
\pagebreak
\tableofcontents
\pagebreak

% Problem 1
\section{Accept-Reject}
Suponga que desea general variables aleatorias de una distribucion $Gamma(\alpha, \beta)$ donde
$\alpha$ no es necesariamente un entero. Decide usar el algoritmo \textbf{Accept-Reject} con la funcion candidata $Gamma(a,b)$.

\subsection{Por que es necesario que $a < \alpha$ y $b> \beta$}

\subsection{Para $a = \left\lfloor \alpha \right\rfloor$, demuestre que $M$ ocurre en $x = \frac{\alpha - \left\lfloor \alpha \right\rfloor}{\frac{1}{\beta} - \frac{1}{b}}$}

\subsection{Para $a = \left\lfloor \alpha \right\rfloor$, encuentre el valor optimo de $b$}

% Problem 2
\section{Implementacion del algoritmo}

\subsection{Describa un algoritmo \textbf{Accept-Reject} para generar una variable aleatoria con distribucion $Gamma(3/2,1)$}


\subsection{Algoritmo en Python}
\subsection{Grafique el histograma de la distribucion obtenida sobreponiendo la distribucion deseada}

\subsection{Estime $E[X^2]$ y construya la grafica de la convergencia de los running means.}

\section{Importance Sampling}

\begin{pythonbox}[title={The function}]
	\inputminted{python}{test.py}
\end{pythonbox}

Usando Importance Sampling estime $E_f\left[ \frac{X^5}{1+(X - 3)^2}I[X \ge 0] \right]$, donde $f$ es la
distribucion $t$ con $v=12$ Utilice las siguientes $g$:
\begin{enumerate}
	\item $Cauchy(0,1)$
	\item $Normal(0, \frac{v}{v-2})$
	\item $Exponencial(\lambda=1)$
\end{enumerate}

\subsection{Estimador importance Sampling}
Para cada una de estas distribuciones presente el estimador que corresponde a la summatoria definida
por el metodo de \textbf{Importance Sampling} y que converge al valor esperado de interes
\subsubsection{$Cauchy(0,1)$}
\subsubsection{$Normal(0, \frac{v}{v-2})$}
\subsubsection{$Exponencial(\lambda=1)$}

\subsection{Estimador Monte Carlo}
Para cada uno presente el estimador que corresponde a la sumatoria definida por el metodo de Integracion
Monte Carlos y que converge al valor esperado de interes.
\subsubsection{$Cauchy(0,1)$}
\subsubsection{$Normal(0, \frac{v}{v-2})$}
\subsubsection{$Exponencial(\lambda=1)$}

\subsection{Implementacion}

\subsection{Graficas}
Construya un asola graica y presente la convergencia de los running menas para los cuatro estimadores. Compare la varianza empirica de los
cuatro estimadores


\end{document}

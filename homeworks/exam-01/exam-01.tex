\documentclass{article}

\usepackage{fancyhdr}
\usepackage{extramarks}
\usepackage{amsmath}
\usepackage{amsthm}
\usepackage{amsfonts}
\usepackage{tikz}
\usepackage[plain]{algorithm}
\usepackage{algpseudocode}
\usepackage{listings}
\usepackage{xcolor}

\lstset{
	language=Python,
	backgroundcolor=\color{black!5}, % Light background for contrast
	basicstyle=\ttfamily\small, % Font style and size
	keywordstyle=\color{blue}\bfseries, % Blue color for keywords
	commentstyle=\color{green!60!black}\itshape, % Green color for comments, italicized
	stringstyle=\color{red}, % Red color for strings
	numbers=left, % Show line numbers on the left side
	numberstyle=\tiny\color{gray}, % Small, gray numbers
	stepnumber=1, % Number every line
	numbersep=5pt, % Distance between line numbers and code
	frame=single, % Adds a frame around the code
	rulecolor=\color{black}, % Frame color
	captionpos=b, % Caption position at the bottom
	breaklines=true, % Line break for long lines
	breakatwhitespace=true, % Break lines at spaces
	showstringspaces=false, % Don't show space as special character
	morekeywords={lambda, with, as} % Add any extra keywords here
}

\usetikzlibrary{automata,positioning}

% Basic Document Settings
\topmargin=-0.45in
\evensidemargin=0in
\oddsidemargin=0in
\textwidth=6.5in
\textheight=9.0in
\headsep=0.25in
\linespread{1.1}

\pagestyle{fancy}
\lhead{\hmwkAuthorName}
\chead{\hmwkClass\ (\hmwkClassInstructor): \hmwkTitle}
\rhead{\firstxmark}
\lfoot{\lastxmark}
\cfoot{\thepage}
\renewcommand\headrulewidth{0.4pt}
\renewcommand\footrulewidth{0.4pt}
\setlength\parindent{0pt}

% Homework Details
\newcommand{\hmwkTitle}{Examen 2}
\newcommand{\hmwkDueDate}{Abril 10, 2025}
\newcommand{\hmwkClass}{ESMA 5015}
\newcommand{\hmwkClassInstructor}{Damaris Santana}
\newcommand{\hmwkAuthorName}{\textbf{Alejandro Ouslan}}

% Title Page
\title{
	\vspace{2in}
	\textmd{\textbf{\hmwkClass:\ \hmwkTitle}}\\
	\normalsize\vspace{0.1in}\small{Due\ on\ \hmwkDueDate}\\
	\vspace{0.1in}\large{\textit{\hmwkClassInstructor}}
	\vspace{3in}
}

\author{\hmwkAuthorName}
\date{}

% New command for subproblems
\newcommand{\subpart}[1]{
	\refstepcounter{subpartCounter}
	\textbf{\large Subproblem \thehomeworkProblemCounter.\thepartCounter.\arabic{subpartCounter}}\quad #1\\
}

\renewcommand{\part}[1]{\textbf{\large Part \thehomeworkProblemCounter.\thepartCounter}\stepcounter{partCounter}\\}

% Begin document
\begin{document}
\maketitle
\pagebreak
\tableofcontents
\pagebreak

% Problem 1
\section{Accept-Reject}
Suponga que desea general variables aleatorias de una distribucion $Gamma(\alpha, \beta)$ donde
$\alpha$ no es necesariamente un entero. Decide usar el algoritmo \textbf{Accept-Reject} con la funcion candidata $Gamma(a,b)$.

\subsection{Por que es necesario que $a < \alpha$ y $b> \beta$}

\subsection{Para $a = \left\lfloor \alpha \right\rfloor$, demuestre que $M$ ocurre en $x = \frac{\alpha - \left\lfloor \alpha \right\rfloor}{\frac{1}{\beta} - \frac{1}{b}}$}

\subsection{Para $a = \left\lfloor \alpha \right\rfloor$, encuentre el valor optimo de $b$}

% Problem 2
\section{Implementacion del algoritmo}

\subsection{Describa un algoritmo \textbf{Accept-Reject} para generar una variable aleatoria con distribucion $Gamma(3/2,1)$}


\subsection{Algoritmo en Python}
\subsection{Grafique el histograma de la distribucion obtenida sobreponiendo la distribucion deseada}

\subsection{Estime $E[X^2]$ y construya la grafica de la convergencia de los running means.}

\section{Importance Sampling}

Usando Importance Sampling estime $E_f\left[ \frac{X^5}{1+(X - 3)^2}I[X \ge 0] \right]$, donde $f$ es la
distribucion $t$ con $v=12$ Utilice las siguientes $g$:
\begin{enumerate}
	\item $Cauchy(0,1)$
	\item $Normal(0, \frac{v}{v-2})$
	\item $Exponencial(\lambda=1)$
\end{enumerate}

\subsection{Estimador importance Sampling}
Para cada una de estas distribuciones presente el estimador que corresponde a la summatoria definida
por el metodo de \textbf{Importance Sampling} y que converge al valor esperado de interes
\subsubsection{$Cauchy(0,1)$}
\subsubsection{$Normal(0, \frac{v}{v-2})$}
\subsubsection{$Exponencial(\lambda=1)$}

\subsection{Estimador Monte Carlo}
Para cada uno presente el estimador que corresponde a la sumatoria definida por el metodo de Integracion
Monte Carlos y que converge al valor esperado de interes.
\subsubsection{$Cauchy(0,1)$}
\subsubsection{$Normal(0, \frac{v}{v-2})$}
\subsubsection{$Exponencial(\lambda=1)$}

\subsection{Implementacion}

\subsection{Graficas}
Construya un asola graica y presente la convergencia de los running menas para los cuatro estimadores. Compare la varianza empirica de los
cuatro estimadores


\end{document}

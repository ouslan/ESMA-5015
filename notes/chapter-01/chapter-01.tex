\documentclass[10pt, oneside]{article}
\usepackage{amsmath, amsthm, amssymb, calrsfs, wasysym, verbatim, bbm, color, graphics, geometry}

\geometry{tmargin=.75in, bmargin=.75in, lmargin=.75in, rmargin = .75in}

\newcommand{\R}{\mathbb{R}}
\newcommand{\C}{\mathbb{C}}
\newcommand{\Z}{\mathbb{Z}}
\newcommand{\N}{\mathbb{N}}
\newcommand{\Q}{\mathbb{Q}}
\newcommand{\Cdot}{\boldsymbol{\cdot}}

\newtheorem{thm}{Theorem}
\newtheorem{defn}{Definition}
\newtheorem{conv}{Convention}
\newtheorem{rem}{Remark}
\newtheorem{lem}{Lemma}
\newtheorem{cor}{Corollary}


\title{ESMA 5015: Simulaciones Estocasticas}
\author{Alejandro Ouslan}
\date{Spring 2025}

\begin{document}

\maketitle
\tableofcontents

\vspace{.25in}

\section{Introduccion}

Sea $X$ una variable aleatoria (V.A) tal que
$$ p(X = i) = \frac{1}{n} \quad \text{para } i = 1, 2, \ldots, n. $$

Calcule $E[X]$
\[
	\begin{split}
		E[X] & = \sum_{i=1}^{n} i \cdot p(X = i)       \\
		& = \sum_{i=1}^{6} i \cdot \frac{1}{6}    \\
		& = \frac{1}{6} \sum_{i=1}^{6} i          \\
		& = \frac{1}{6} \cdot \frac{6 \cdot 7}{2} \\
		& = \frac{7}{2} = 3.5
	\end{split}
\]

para generar alores de un avariable aleatoria se usa sample(1:6, observasiones, replace = TRUE). Detras del sample hay una distribucion uniforme.
El generador aleatorio es bueno si internamente genera buenas variables uniformes.

\section{Distribusion empirica e probabilidad}

$t = 1, 2, \ldots, n$


\section{Simulacion de insectos}

Un insecto produce un gran numero de huevos y cada uno sobrevive con probabilidad $p$. en promedio. Cuantos huevos sobreviven?

\subsection{Metodo 1}

\begin{enumerate}
	\item variable aleatoria $x = \text{numero de huevos}$ y $x \sim \text{Binomial}(n, p)$
	\item variable aleatoria $y= \text{un gran numero de huevos}$ y $y \sim \text{Poisson}(\lambda)$
\end{enumerate}

Si $x$ y $y$ son variables aleatorias, entonces $E[x] = E[E[x|y]]$

\[
	\begin{split}
		E[x|y] = yp \\
		E[x] = E[yp] = pE[y] = p\lambda \\
	\end{split}
\]

\subsection{Metodo 2}





\end{document}

\documentclass[10pt, oneside]{article}
\usepackage{amsmath, amsthm, amssymb, calrsfs, wasysym, verbatim, bbm, color, graphics, geometry}

\geometry{tmargin=.75in, bmargin=.75in, lmargin=.75in, rmargin = .75in}

\newcommand{\R}{\mathbb{R}}
\newcommand{\C}{\mathbb{C}}
\newcommand{\Z}{\mathbb{Z}}
\newcommand{\N}{\mathbb{N}}
\newcommand{\Q}{\mathbb{Q}}
\newcommand{\Cdot}{\boldsymbol{\cdot}}

\newtheorem{thm}{Theorem}
\newtheorem{defn}{Definition}
\newtheorem{conv}{Convention}
\newtheorem{rem}{Remark}
\newtheorem{lem}{Lemma}
\newtheorem{cor}{Corollary}


\title{ESMA 5015: Simulaciones Estocasticas}
\author{Alejandro Ouslan}
\date{Spring 2025}

\begin{document}

\maketitle
\tableofcontents

\vspace{.25in}

\section{Introduccion}

Sea $X$ una variable aleatoria (V.A) tal que
$$ p(X = i) = \frac{1}{n} \quad \text{para } i = 1, 2, \ldots, n. $$

Calcule $E[X]$
\[
	\begin{split}
		E[X] & = \sum_{i=1}^{n} i \cdot p(X = i)       \\
		     & = \sum_{i=1}^{6} i \cdot \frac{1}{6}    \\
		     & = \frac{1}{6} \sum_{i=1}^{6} i          \\
		     & = \frac{1}{6} \cdot \frac{6 \cdot 7}{2} \\
		     & = \frac{7}{2} = 3.5
	\end{split}
\]
\end{document}
